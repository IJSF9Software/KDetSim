%
% A simple LaTeX template for Books
%  (c) Aleksander Morgado <aleksander@es.gnu.org>
%  Released into public domain
%

\documentclass{book}
\usepackage[a4paper, top=3cm, bottom=3cm]{geometry}
\usepackage[latin1]{inputenc}
\usepackage{setspace}
\usepackage{fancyhdr}
\usepackage{tocloft}



\begin{document}


\pagestyle{empty}
%\pagenumbering{}
% Set book title
\title{\textbf{KDetSim - a simple way to simulate detectors}}
% Include Author name and Copyright holder name
\author{Gregor Kramberger, Jo\v zef Stefan Institute, Slovenia}



% 1st page for the Title
%-------------------------------------------------------------------------------
\maketitle


% 2nd page, thanks message
%-------------------------------------------------------------------------------
\thispagestyle{empty}
\thanks{Thanks to many colleaques who helped to make it happen.}
\newpage



% General definitions for all Chapters
%-------------------------------------------------------------------------------

% Define Page style for all chapters
\pagestyle{fancy}
% Delete the current section for header and footer
\fancyhf{}
% Set custom header
\lhead[]{\thepage}
\rhead[\thepage]{}

% Set arabic (1,2,3...) page numbering
\pagenumbering{arabic}

% Set double spacing for the text
\doublespacing



% Not enumerated chapter
%-------------------------------------------------------------------------------
\chapter*{Preface}

The semiconductor particle detectors are a must in most of modern high energy
experiments, X-rays and synchrotron imaging application, Space missions etc. 



% If the chapter ends in an odd page, you may want to skip having the page
%  number in the empty page
\newpage
\thispagestyle{empty}



% First enumerated chapter
%-------------------------------------------------------------------------------
\chapter{Introduction}

The state of the art TCAD tools for simulation of semiconductor device properties 
are vital for successful device development and process simulation at different 
semiconductor detector processing/designing facilities. They are commercialy developed
for markets of which high energy physics community is not the most important part. On the other 
hand the requirements for high energy physics application of these tools are in a sense 
unique and not so well covered. 

Application of semiconductor detectors in high radiation 
fields invevitably leads to different levels of radiation damage. Althogh many efforts have
been spent in the past to simulate the effects of deep traps to detector operation, 
at present we don't know the set of traps that would entirely reproduce the changes in 
device operation properties, such as electric field profile, charge collection efficiency and leakage 
current. A single trap is defined by 6 parameters (electron,hole capture cross-sections, 
energy level, concentration and type). Moreover the SRH may not be the only physics that governes the
device behavior. Accuracy of models such as trapp-assisted tunneling in numerical calculations 
not only adds to complexity, but also renders the convergence of equation solver problematic.

TCAD tools are also not well suited for Monte Carlo simulation, bread and butter of 
high energy physics simulations. They are ofter required for e.g. detection efficiency studies, 
position resolution etc. Although with some effort one can generate different initial 
charge distribution deposited by ionizing particle or light pulse, it is very demanding to simulate 
large multi-electrode structures.
 

\chapter{Physics basics}
Mama mia

\chapter{Calculation of the field}
adsad
\chapter{Classes}
sdasda
\chapter{KDetector}
dasdas
\chapter{Examples}

% Last pages for ToC
%-------------------------------------------------------------------------------
\newpage
% Include dots between chapter name and page number
\renewcommand{\cftchapdotsep}{\cftdotsep}
%Finally, include the ToC
\tableofcontents

\end{document}


